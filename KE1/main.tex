\documentclass{article}
\usepackage{amssymb}
\usepackage{amsmath}
\setlength{\parindent}{0pt}
\title{Mathematische Grundlagen KE1 Einsendeaufgaben}
\author{Alexander Seidmann}
\begin{document}
\maketitle{}
\setcounter{section}{1}
\subsection{}
\subsubsection{}
Behauptung: $A\Rightarrow (B\Rightarrow C)$ und $(A\land B)\Rightarrow C$ sind logisch äquivalent.

\begin{center}
    \begin{tabular}{|c|c|c|c|c|}
        \hline
        A & B & C & $B\Rightarrow C$ & $(A\Rightarrow (B\Rightarrow C))$ \\
        \hline
        w & w & w & w & w \\
        w & w & f & f & f \\
        w & f & w & w & w \\
        w & f & f & w & w \\
        f & w & w & w & w \\
        f & w & f & f & w \\
        f & f & w & w & w \\
        f & f & f & w & w \\
        \hline
    \end{tabular}
\end{center}

\begin{center}
    \begin{tabular}{|c|c|c|c|c|}
        \hline
        A & B & C & $A\land B$ & $(A\land B)\Rightarrow C$ \\
        \hline
        w & w & w & w & w \\
        w & w & f & w & f \\
        w & f & w & f & w \\
        w & f & f & f & w \\
        f & w & w & f & w \\
        f & w & f & f & w \\
        f & f & w & f & w \\
        f & f & f & f & w \\
        \hline
    \end{tabular}
\end{center}

$A\Rightarrow (B\Rightarrow C)$ und $(A\land B)\Rightarrow C$ sind logisch äquivalent.
$\square{}$

\subsubsection{}
Behauptung: $(A\Rightarrow B)\Rightarrow C$ und $((\lnot{A})\Rightarrow C)\land (B\Rightarrow C)$ sind logisch äquivalent.

\begin{center}
    \begin{tabular}{|c|c|c|c|c|c|}
        \hline
        A & B & C & $(\lnot A)\Rightarrow C$ & $B\Rightarrow C$ & $((\lnot A)\Rightarrow C)\land(B\Rightarrow C)$ \\
        \hline
        w & w & w & w & w & w \\
        w & w & f & w & f & f \\
        w & f & w & w & w & w \\
        w & f & f & w & w & w \\
        f & w & w & w & w & w \\
        f & w & f & f & f & f \\
        f & f & w & w & w & w \\
        f & f & f & f & w & f \\
        \hline
    \end{tabular}
\end{center}

\begin{center}
    \begin{tabular}{|c|c|c|c|c|}
        \hline
        A & B & C & $A\Rightarrow B$ & $(A\Rightarrow B)\Rightarrow C$ \\
        \hline
        w & w & w & w & w \\
        w & w & f & w & f \\
        w & f & w & f & w \\
        w & f & f & f & w \\
        f & w & w & w & w \\
        f & w & f & w & f \\
        f & f & w & w & w \\
        f & f & f & w & f \\
        \hline
    \end{tabular}
\end{center}
$(A\Rightarrow B)\Rightarrow C$ und $((\lnot{A})\Rightarrow C)\land (B\Rightarrow C)$ sind logisch äquivalent.
$\square{}$

\subsection{}
\subsubsection{}
Behauptung: Es gilt $n^2>n+1$ für alle $n\geq 2$.

Induktionsanfang:

$$n=2$$
$$2^2>2+1$$
$$4>3$$
Es gilt der Induktionsanfang.

Induktionsbehauptung:
$$(n+1)^2>(n+1)+1$$

Beweis:
$$(n+1)^2>n+1+(((n+1)+1)-(n+1))$$
$$(n+1)^2>n+1+1$$
$$(n+1)^2>(n+1)+1$$

Mit dem Prinzip der Vollständigen Induktion folgt für alle $n\in \mathbf{N}$ mit $n \geq 2$ gilt $n^2>n+1$.
$\square$

\subsubsection{}
Behauptung: Es gilt $n^2\geq2n+3$ für alle $n\geq 3$.

Induktionsanfang:
$$n=3$$
$$3^2\geq 2*3+3$$
$$9\geq 9$$
Es gilt der Induktionsanfang.

Induktionsschritt:
$$(n+1)^2\geq 2(n+1)+3$$

Beweis:
$$(n+1)^2\geq 2n+3+((2(n+1)+3)-(2n+3))$$
$$(n+1)^2\geq 2n+3+2$$
$$(n+1)^2\geq 2n+2+3$$
$$(n+1)^2\geq 2(n+1)+3$$

Mit dem Prinzip der Vollständigen Induktion folgt: für alle $n\in \mathbf{N}$ mit $n \geq 3$ gilt $n^2\geq2n+3$.
$\square$

\subsection{}
\subsubsection{}

$$
X = \begin{pmatrix}
    2 & 2 & 2 \\
    -1 & -1 & -1 
\end{pmatrix}
$$

\subsubsection{}
$$
AB = \begin{pmatrix}
    -1 & -8 & -10 \\
    1 & -2 & -5 \\ 
    9 & 22 & 15 
\end{pmatrix}
$$
$$
BA = \begin{pmatrix}
    15 & -21 \\
    10 & -3 
\end{pmatrix}
$$

\subsection{}
\subsubsection{}
$$f: N\rightarrow N$$
$$
f(x)=
\begin{cases}
    1, \text{wenn } x \text{ ungerade} \\
    \frac{x}{2}, \text{wenn } x \text{ gerade} 
\end{cases}
$$

(a) ist erfüllt, da $\frac{x}{2}$ alle Zahlen $\mathbf{N}$ abbildet.

(b) ist erfüllt, da die Urbilder von $f(1)$ alle ungeraden Zahlen in $\mathbf{N}$ sind.

\subsubsection{}
$$g: N\rightarrow N$$
$$g(x)=2x$$

(a) ist erfüllt, da es für jedes Element $g(x)$ genau ein Urbild $x\in \mathbf{N}$ gibt. 

(b) ist erfüllt, da $Bild(g)$ alle geraden Zahlen in $\mathbf{N}$ sind, und dadurch
$\mathbf{N}\not Bild(g)$ alle ungeraden Zahlen sind, also eine unendlich große Menge.

\subsection{}
Sei $A\in M_{mn}(\mathbf{K})$ eine Matrix, sodass $XA=0\in M_{mn}(\mathbf{K})$ für alle Matrizen $X\in M_{mm}(K)$ gilt.
Beweisen Sie, dass A die Nullmatrix in $M_{mn}(K)$ ist. \\
Annahme:
A ist die Nullmatrix in $M_{mn}(K)$.
Sei $m=2$ und $n=3$ und X somit eine Matrix $
\begin{pmatrix}
    a & b \\
    c & d
\end{pmatrix}
$ und A eine Matrix $
\begin{pmatrix}
    e & f & g \\
    h & i & j
\end{pmatrix}
$. 

\vspace{10mm}

Dann ist $XA=
\begin{pmatrix}
    e & f & g \\
    h & i & j
\end{pmatrix}
=
\begin{pmatrix}
    $ae+bh$ & $af+bi$ & $ag+bj$ \\
    $ce+dh$ & $cf+di$ & $cg+dj$
\end{pmatrix}
$. 

\vspace{10mm}

Da XA die Nullmatrix in $M_{mn}(K)$ ist, muss 
$$
\begin{pmatrix}
    $ae+bh=0$ & $af+bi=0$ & $ag+bj=0$ \\
    $ce+dh=0$ & $cf+di=0$ & $cg+dj=0$
\end{pmatrix}
$$ gelten. \\
Sei nun $XA = (z_{ij})$, so muss sowohl der linke als auch der rechte Summand in Berechnung von $z_{ij}$ immer 0 sein, oder die Summanden invers zueinander sein. \\
Da es keine Zahlen $m, n\in K\setminus 0$ gibt, sodass für jede Zahlenkombination $x, y\in K\setminus 0$ $mx + ny = 0$ ist, muss A die Nullmatrix sein. $\square$


\end{document}
